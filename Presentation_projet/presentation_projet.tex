\documentclass{report}
\usepackage[utf8]{inputenc} % force the use of utf8
\usepackage[T1]{fontenc} % font encoding, allows accents
\usepackage[papersize={21cm,29.7cm},top= 2.5cm,bottom=2.5cm, inner=2.5cm, outer=2.5cm]{geometry} % page formatting
\usepackage{graphicx} % images management
\usepackage{wrapfig} % floating images
\usepackage{array} % allow arrays
\usepackage{fancyhdr} % headers/footers management (overrides empty, plain and headings)
\usepackage{listings} % code insertion (MUST BE WRITTEN AFTER BABEL)
\usepackage{enumitem} % for /setlist
\usepackage{color,soul} % add some colors and highlight
\usepackage{xcolor} % more colors
\usepackage{float}
\usepackage{bm}
\usepackage{amsmath}
\usepackage[hyphens]{url} % auto break lines in URL
\usepackage[toc,page]{appendix}
\usepackage{titlesec}
\usepackage[hidelinks,  colorlinks  = true, % no borders, colors enabled
                        anchorcolor = blue,
                        linkcolor   = black, % links in table of contents
                        urlcolor    = blue,
                        citecolor   = blue]{hyperref}


\sethlcolor{cyan} % package soul
\newcommand{\file}[1]{\hl{\emph{#1}}} % highlight a file URI

\graphicspath{{Resources/}}

%%%%%%%%%%%%%%%%%%%%%%%%%%%%%%%%%%%%%%%%%%%%%%%%%%%%%%%% LISTINGS %%%%%%%%%%%%%%
\definecolor{comment}{rgb}{0.12, 0.38, 0.18 }
\definecolor{keyword}{rgb}{0.37, 0.08, 0.25}  % #5F1441
\definecolor{string}{rgb}{0.06, 0.10, 0.98} % #101AF9
%%%%%%%%%%%%%%%%%%%%%%%%%%%%%%%%%%%%%%%%%%%%%%%%%%%%%%%%%%%%%%%%%%%%%%%%%%%%%%%%

\newcommand\sectionpostlude{
  \vspace{0.8em}
}
\setlength{\intextsep}{10mm}
\fancypagestyle{plain}{
    %---------------------------------------------------------------------------
    % HEADER
    %---------------------------------------------------------------------------
    \fancyhead[R]{Apprentissage et reconnaissance - Projet}

    %---------------------------------------------------------------------------
    % FOOTER
    %---------------------------------------------------------------------------
    \renewcommand{\footrulewidth}{0.1pt}
    \fancyfoot[C]{Baptiste AMATO, Arnoud VANHUELE \& Alexandre CHAVENON}
    \fancyfoot[LE]{\ifnum\thepage>0 \thepage \fi}
    \fancyfoot[RO]{\ifnum\thepage>0 \thepage \fi}
}

\fancypagestyle{empty}{%
    \renewcommand{\headrulewidth}{0pt} % No sub line
    \fancyhead{} % Empty the header

    \renewcommand{\footrulewidth}{0pt}
    \fancyfoot{}
}

\setlist[itemize,2]{label={$\bullet$}} % use bullets for nested itemize

% First page
\newcommand{\presentation}[1]{\vspace{0.3cm}\large{\textbf{#1}}\vspace{0.3cm}\\}
\newcommand{\presentationLarge}[1]{\vspace{0.3cm}\LARGE{\textbf{#1}}\vspace{0.3cm}\\}

% Overrides chapter (numbered and no-numbered) headings: remove space, display only the title
\makeatletter
  \def\@makechapterhead#1{%
  \vspace*{0\p@}% avant 50
  {\parindent \z@ \raggedright \normalfont
    \interlinepenalty\@M
    \Huge \bfseries \thechapter\quad #1
    \vskip 40\p@
  }}
  \def\@makeschapterhead#1{%
  \vspace*{0\p@}% before 50
  {\parindent \z@ \raggedright
    \normalfont
    \interlinepenalty\@M
    \Huge \bfseries  #1\par\nobreak
    \vskip 40\p@
  }}
\makeatother

\newcommand{\ignore}[1]{} % inline comments

\pagenumbering{arabic}
\pagestyle{plain} % uses fancy

%%%%%%%%%%%%%%%%%%%%%%%%%%%%%%%%%%%%%%%%%%%%%%%%%%%%%%%%%%%%%%%%%%%%%%%%%%%%%%%%

%-------------------------------------------------------------------------------
% DOCUMENT INFO SECTION
%-------------------------------------------------------------------------------
\title{Apprentissage et reconnaissance - GIF-4101/GIF-7005 - Projet}
\author{Baptiste AMATO, Arnoud VANHUELE \& Alexandre CHAVENON}
\date\today

\begin{document}
\thispagestyle{empty} % only for the current page

\newcommand{\HRule}{\rule{\linewidth}{0.5mm}} % Defines a new command for the horizontal lines, change thickness here

\begin{center}
\vspace{2.5cm}
\presentation{Université LAVAL}

%-------------------------------------------------------------------------------
% TITLE SECTION
%-------------------------------------------------------------------------------

\vspace{4cm}
\noindent{
\begin{minipage}{0.9\textwidth}
\begin{center}
  \HRule \\[0.4cm]
  { \huge \bfseries Apprentissage et reconnaissance \\ GIF-4101/GIF-7005}\\[0.4cm] % Title of the document
  { Projet : Détection automatique de prolongements neuronaux }\\ % Sub-Title of the document
  \HRule \\[1.5cm]
\end{center}
\end{minipage}}
\vspace{4cm}


%-------------------------------------------------------------------------------
% AUTHOR SECTION
%-------------------------------------------------------------------------------

\begin{minipage}{0.4\textwidth}
\begin{flushleft} \large
\emph{Auteurs :}\\
Baptiste \textsc{Amato} \\
Arnoud \textsc{Vanhuele} \\
Alexandre \textsc{Chavenon} \\
\end{flushleft}
\end{minipage}

\end{center}

%%%%%%%%%%%%%%%%%%%%%%%%%%%%%%%%%%%%%%%%%%%%%%%%%%%%%%%%%%%%%%%%%%%%%%%%%%%%%%%%

\chapter{Introduction}

\section{Présentation du projet}

Le projet est proposé par le centre de recherche CERVO. Il consiste à mettre en
place un réseau de neurones permettant de reconnaitre des axones et des dendrites
sur des images d’une protéine, en étiquetant ces images n’ayant pas de marqueurs
axonaux et dendritiques.

\section{Jeu de données}

Le jeu de données initial comprend 1024 images au format \textit{.tiff}, ayant
chacune 3 canaux : un pour l'actine (la protéine d'intérêt), un pour les axones,
et un pour les dendrites. \\
Ce jeu de données étant relativement petit pour un apprentissage par réseau neuronal,
nous allons utiliser des méthodes d'augmentation comme les symétries, rotations, ou
encore découpes de sous-parties des images.


\end{document}

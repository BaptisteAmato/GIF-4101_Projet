\documentclass{report}
\usepackage[utf8]{inputenc} % force the use of utf8
\usepackage[T1]{fontenc} % font encoding, allows accents
\usepackage[papersize={21cm,29.7cm},top= 2.5cm,bottom=2.5cm, inner=2.5cm, outer=2.5cm]{geometry} % page formatting
\usepackage{graphicx} % images management
\usepackage{wrapfig} % floating images
\usepackage{array} % allow arrays
\usepackage{fancyhdr} % headers/footers management (overrides empty, plain and headings)
\usepackage{listings} % code insertion (MUST BE WRITTEN AFTER BABEL)
\usepackage{enumitem} % for /setlist
\usepackage{color,soul} % add some colors and highlight
\usepackage{xcolor} % more colors
\usepackage{float}
\usepackage{bm}
\usepackage{amsmath}
\usepackage[hyphens]{url} % auto break lines in URL
\usepackage[toc,page]{appendix}
\usepackage{titlesec}
\usepackage[hidelinks,  colorlinks  = true, % no borders, colors enabled
                        anchorcolor = blue,
                        linkcolor   = black, % links in table of contents
                        urlcolor    = blue,
                        citecolor   = blue]{hyperref}


\sethlcolor{cyan} % package soul
\newcommand{\file}[1]{\hl{\emph{#1}}} % highlight a file URI

\graphicspath{{Resources/}}

%%%%%%%%%%%%%%%%%%%%%%%%%%%%%%%%%%%%%%%%%%%%%%%%%%%%%%%% LISTINGS %%%%%%%%%%%%%%
\definecolor{comment}{rgb}{0.12, 0.38, 0.18 }
\definecolor{keyword}{rgb}{0.37, 0.08, 0.25}  % #5F1441
\definecolor{string}{rgb}{0.06, 0.10, 0.98} % #101AF9
%%%%%%%%%%%%%%%%%%%%%%%%%%%%%%%%%%%%%%%%%%%%%%%%%%%%%%%%%%%%%%%%%%%%%%%%%%%%%%%%

\newcommand\sectionpostlude{
  \vspace{0.8em}
}
\setlength{\intextsep}{10mm}
\fancypagestyle{plain}{
    %---------------------------------------------------------------------------
    % HEADER
    %---------------------------------------------------------------------------
    \fancyhead[R]{Apprentissage et reconnaissance - Projet}

    %---------------------------------------------------------------------------
    % FOOTER
    %---------------------------------------------------------------------------
    \renewcommand{\footrulewidth}{0.1pt}
    \fancyfoot[C]{Baptiste AMATO, Alexandre CHAVENON \& Arnoud VANHUELE}
    \fancyfoot[LE]{\ifnum\thepage>0 \thepage \fi}
    \fancyfoot[RO]{\ifnum\thepage>0 \thepage \fi}
}

\fancypagestyle{empty}{%
    \renewcommand{\headrulewidth}{0pt} % No sub line
    \fancyhead{} % Empty the header

    \renewcommand{\footrulewidth}{0pt}
    \fancyfoot{}
}

\setlist[itemize,2]{label={$\bullet$}} % use bullets for nested itemize

% First page
\newcommand{\presentation}[1]{\vspace{0.3cm}\large{\textbf{#1}}\vspace{0.3cm}\\}
\newcommand{\presentationLarge}[1]{\vspace{0.3cm}\LARGE{\textbf{#1}}\vspace{0.3cm}\\}

% Overrides chapter (numbered and no-numbered) headings: remove space, display only the title
\makeatletter
  \def\@makechapterhead#1{%
  \vspace*{0\p@}% avant 50
  {\parindent \z@ \raggedright \normalfont
    \interlinepenalty\@M
    \Huge \bfseries \thechapter\quad #1
    \vskip 40\p@
  }}
  \def\@makeschapterhead#1{%
  \vspace*{0\p@}% before 50
  {\parindent \z@ \raggedright
    \normalfont
    \interlinepenalty\@M
    \Huge \bfseries  #1\par\nobreak
    \vskip 40\p@
  }}
\makeatother

\newcommand{\ignore}[1]{} % inline comments

\pagenumbering{arabic}
\pagestyle{plain} % uses fancy

%%%%%%%%%%%%%%%%%%%%%%%%%%%%%%%%%%%%%%%%%%%%%%%%%%%%%%%%%%%%%%%%%%%%%%%%%%%%%%%%

%-------------------------------------------------------------------------------
% DOCUMENT INFO SECTION
%-------------------------------------------------------------------------------
\title{Apprentissage et reconnaissance - GIF-4101/GIF-7005 - Projet}
\author{Baptiste AMATO, Alexandre CHAVENON \& Arnoud VANHUELE}
\date\today

\begin{document}
\thispagestyle{empty} % only for the current page

\newcommand{\HRule}{\rule{\linewidth}{0.5mm}} % Defines a new command for the horizontal lines, change thickness here

\begin{center}
 \vspace{2.5cm}
 \presentation{Université LAVAL}

 %-------------------------------------------------------------------------------
 % TITLE SECTION
 %-------------------------------------------------------------------------------

 \vspace{4cm}
 \noindent{
  \begin{minipage}{0.9\textwidth}
   \begin{center}
    \HRule \\[0.4cm]
    { \huge \bfseries Apprentissage et reconnaissance \\ GIF-4101/GIF-7005}\\[0.4cm] % Title of the document
    { Projet : Détection automatique de prolongements neuronaux }\\ % Sub-Title of the document
    \HRule \\[1.5cm]
   \end{center}
  \end{minipage}}
 \vspace{4cm}


 %-------------------------------------------------------------------------------
 % AUTHOR SECTION
 %-------------------------------------------------------------------------------

 \begin{minipage}{0.4\textwidth}
  \begin{flushleft} \large
   \emph{Auteurs :}\\
   Baptiste \textsc{Amato} \\
   Alexandre \textsc{Chavenon} \\
   Arnoud \textsc{Vanhuele} \\
  \end{flushleft}
 \end{minipage}

\end{center}

%%%%%%%%%%%%%%%%%%%%%%%%%%%%%%%%%%%%%%%%%%%%%%%%%%%%%%%%%%%%%%%%%%%%%%%%%%%%%%%%

\chapter{Introduction}

\section{Présentation du projet}

Le projet est proposé par le centre de recherche CERVO.  Il consiste à reconnaître
des axones et des dendrites sur des images d’une protéine (actine), en étiquetant ces images
n’ayant pas de marqueurs axonaux et dendritiques. Nous disposons d’une banque de
données d’images déjà marquées : il s'agit donc un problème d’apprentissage supervisé.

\section{Jeu de données}

Le jeu de données initial comprend 1024 images au format \textit{.tiff}, ayant
chacune 3 canaux : un pour l'actine (la protéine d'intérêt), un pour les axones,
et un pour les dendrites. \\
Ce jeu de données étant relativement petit pour un apprentissage par réseau neuronal,
nous allons utiliser des méthodes d'augmentation comme les symétries, rotations, ou
encore découpes de sous-parties des images.

\section{Etat de l'art}

Il s'agit ici de détecter différents objets dans une image (axones et dendrites
à partir d'une image globale d'actine) : c'est un problème de détection particulier,
car il n'est pas possible d'encadrer les objets par des "bounding boxes", utilisées
par exemple pour la détection de visage, de personnes ou de voitures ; on cherche
alors à détecter le contour des objets. Un article de recherche assez récent a
démontré une capacité de détection de contour impressionnante : \textit{Object Contour
 Detection with a Fully Convolutional Encoder-Decoder Network}, par \textbf{Yang
 \textit{et al.}}. Nous pensons donc nous orienter vers un réseau de neurones profond
avec une architecture \textit{Encoder-Decoder} ; cette architecture est aussi
utilisé dans les traductions de textes (séquences en entrée et sortie). Des résultats
probants concernant de la segmentation d'images sont présentés dans l'article
\textit{Iterative Deep Convolutional Encoder-Decoder Network for Medical Image
 Segmentation}, par \textbf{Jung Uk Kim, Hak Gu Kim, et Yong Man Ro}, suivant une
architecture similaire (\textit{Encoder-Decoder}). \\
Les principes de segmentation d'image sont clairement expliqués dans l'article
\textbf{Fully Convolutional Networks for Semantic Segmentation}, par \textbf{Shelhamer
 \textit{et al.}}. \\
Concernant l’augmentation de notre jeu de données, nous aurons une approche
classique par traitement d’image traditionnel et dans un second temps, nous
explorerons la possibilité d'utiliser un Generative Adversarial Network pour l’augmentation
des données, comme précisé dans l'article\textit{Biomedical Data Augmentation Using
Generative Adversarial Neural Networks. Artificial Neural Networks and Machine
Learning} par \textbf{Calimeri, F.
 \textit{et al.}}.


%%%%%%%%%%%%%%%%%%%%%%%%%%%%%%%%%%%%%%%%%%%%%%%%%%%%%%%%%%%%%%%%%%%%%
\chapter{Pré-traitements}

\section{Masques}

Une image du jeu de données fournit contient trois canaux, pour l'actine, les axones
et les dendrites. Nous séparons donc le \textit{.tiff} afin d'obtenir trois images,
avec l'actine en vert, les axones en rougeet les dendrites en bleu. Après traitements,
on a des images comme suit :

\begin{figure}[H]
\centering
\includegraphics[scale=0.35]{"ex_mask"}
\caption{En haut, un triplet actine-axone-dendrite, et en bas les mêmes images une fois le masque appliqué}
\end{figure}

Le pré-traitement commence par une égalisation de l'histogramme des intensités
des images en niveaux de gris. On utilise ensuite deux opérations, une d'érosion
afin de ne garder que du contenu autour des contours, et une de flou gaussien,
afin d'obtenir des contours légèrement plus denses. \\
On applique ensuite un \textit{thresholding} sur chaque image,
permettant de supprimer du bruit, soit les pixels de très faible intensité :
on définit un seuil entre 0 et 255 (dans notre cas, 10), et tous les pixels ayant
une valeur inférieure à 10 sont ramenés à 0. On convertit ensuite les images en
RGB, et on ne conserve des valeurs non nulles que pour un canal par image(vert pour
l'actine, rouge pour les axones et bleu pour les dendrites).


\section{Images d'entrée}

\subsection{Taille}

Les images en entrée sont toutes de dimensions différentes, mais de même résolution.
Il est donc nécessaire de conserver ces résolutions (ayant un sens physique) en
n'effectuant aucun redimmensionnement. Le problème est que le réseau de neurones
attend des images de même taille en entrée : nous avons donc découpé chacune des
images en \textit{crops} (petits carrés) de taille 224x224, correspondant à la taille
des images en entrée du réseau \textbf{VGG16}, mentionné plus bas.

\subsection{Contenu}

Notre problème est une détection de contour, mais nous devons inclure la notion
d'\textbf{intensité} dans nos matrices d'entrée (valeur allant de 0 à 255).
Les images sont normalisées entre 0 et 1 ; une matrice d'entraînement sera de
taille 224x224x1 (intensités de l'actine), et le label associé sera une matrice
de taille 224x224x2, avec le premier channel correspondant aux axones et le second
aux dendrites.

\section{Prédiction d'une nouvelle image}

Là encore, il est nécessaire de procéder par \textit{crops} pour traiter l'image
à traiter. Ainsi, chaque \textit{crop} sera associé à une image de prédiction,
et l'image résultante sera reconstituée à partir de ces \textit{crops}. \\
On décompose en deux images la prédiction reçue, afin d'obtenir le masque des axones
et celui des dendrites, qui l'on peut alors supperposer à l'image d'actine originale.


%%%%%%%%%%%%%%%%%%%%%%%%%%%%%%%%%%%%%%%%%%%%%%%%%%%%%%%%%%%%%%%%%%%%%
\chapter{Réseau de neurones}

Nous utilisons la librairie \textit{Keras} avec un back-end en \textit{TensorFlow}.
L'architecture sera celle présentée dans l'article de recherche \textit{Object Contour
Detection with a Fully Convolutional Encoder-Decoder Network}, par \textbf{Yang
\textit{et al.}} : la première parte du réseau est l'encodeur, basé sur l'architecture
du réseau \textbf{VGG16}, en s'arrêtant juste avant le \textit{Fully Connected layer}.
Ensuite, le décodeur est celui décrit par l'article, permettant de reconstituer
une image de la taille d'origine avec des opérations de \textit{déconvolution},
qui en \textit{Keras} se font grâce à l'opération \textit{UpSampling} suivie d'une
\textit{same convolution} (convolution ne modifiant pas la taille de l'image d'entrée
grâce à l'ajout de \textit{padding}). \\
L'optimisation est faite avec la méthode \textbf{Adam} utilisant les avantages des
algorithmes \textbf{AdaGrad} et \textbf{RMSProp}  ; celle-ci a fait ses preuves
dans le monde du Deep Learning.
+ TODO: loss function



\end{document}
